\documentclass[a4paper,12pt]{article}

\usepackage{enumitem}
\usepackage{marvosym}
\usepackage{tabularx} % fancy tables
\usepackage[margin=1cm]{geometry}
\usepackage{fontspec} % for loading fonts
\usepackage{xunicode,xltxtra,url,parskip} % other packages for formatting
\RequirePackage{color,graphicx}
\usepackage[usenames,dvipsnames]{xcolor}
\usepackage[big]{layaureo} % better formatting of the A4 page
\usepackage{titlesec} % custom \section

% Setup hyperref package, and colours for links
\usepackage{hyperref}
\definecolor{linkcolour}{rgb}{0,0.2,0.6}
\hypersetup{colorlinks,breaklinks,urlcolor=linkcolour, linkcolor=linkcolour}

\titleformat{\section}{\large\scshape\bfseries}{}{0em}{}[\titlerule]
\titlespacing{\section}{0pt}{3pt}{3pt}
% Tweak a bit the top margin
\addtolength{\voffset}{-1.0cm}
% Tweak a bit the left margin
\addtolength{\oddsidemargin}{-1.0cm}


%-------------WATERMARK TEST [**not part of a CV**]---------------
\usepackage[absolute]{textpos}

\setlength{\TPHorizModule}{30mm}
\setlength{\TPVertModule}{\TPHorizModule}
\textblockorigin{2mm}{0.65\paperheight}
\setlength{\parindent}{0pt}

%--------------------BEGIN DOCUMENT----------------------
\begin{document}

\pagestyle{empty} % non-numbered pages

%--------------------Header-------------

% Name
\Huge Austin Jones\smallskip{} \\
% Address
\small \textsc{Address:} 201 S Central St, Knoxville, TN \\
% Phone
\small \textsc{Phone:} 615\--962\--3732 \\
% Email
\textsc{Email:} \href{mailto:ajones53.aj@gmail.com}{ajones53.aj@gmail.com} \\
% Github
\textsc{GitHub:} \href{https://github.com/ajone239}{https://github.com/ajone239} \\

% Today
\today

% Salutation
Dear Hiring Manager,

% Introduction
I am writing you to apply to the Software Engineering position at NexHealth.
I've been working in industry for about 2 years now, but I've nearly 5 years of engineering experience given the work I did in internships.
Working with distributed systems to provide a scalable and reliable service to a global customer base is something that has always interested me about cloud development.


% Body paragraph
At the University of Tennessee Knoxville, I gained a fundamental basis of software and the best practices of Software Development.
These practices were further reinforced in my internships.
At Siemens, I worked on many projects ranging up and down the stack.
From GUI test applications to embedded firmware, I had my hands in many long term projects; some of which are still in use.
Consequently, I was also responsible for presenting and communicating my work/findings, requiring effective understanding and organization of ideas to show to audiences of varying technicality.
At Garmin, my efforts were more focused.
In a summer, I became very acquainted with a particular microcontroller and used it to write the power-on logic for a new product.
My sole responsibilities were to realize this software, but it involved communicating with the rest of the team, reading schematics, and much code review to inform firmware decisions.

My time in industry was spent on much higher level software.
At Polysign, I worked almost exclusively on cloud-based, distributed applications.
As it was a start-up, I was often handed the reigns on designs or even full features.
The fast and agile development process allowed for whole top-level designs to be usurped for better ones.
Team cross talk and integration between teams wasn't common but vital to the bringing up features, products, or debugging production incidents.
Technologies across the firms were new and abundant meaning that picking up a small task may also involve learning two or three new technologies and how to use them.
This kept me on my toes for entering new projects, expecting nothing and turning to testing to get my footing.

% Closing paragraph
Thank you for taking the time to read my application and considering me for this position.
If you've any questions, feel free to reach me with the information above.
I'm sure my mixture of skills and work ethic could be a great fit for this role.

% Letter ending and signature
Best, \\
Austin Jones

\end{document}
